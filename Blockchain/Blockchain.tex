\documentclass[a4paper, 12pt]{report}
\usepackage{amsmath}
\usepackage{graphicx}
\usepackage{amssymb}
\usepackage{tikz}

\newtheorem{theorem}{Theorem}

\begin{document}
\chapter{Introduction}
\section{Blockchain definition}
Blockchain is a list of records, called \textit{blocks}, that are securely linked together
using cryptograhpy. Each block contains a cryptograhpic hash of the previous block, a \textit{timestamp}
, and transaction data, generally represented as a \textit{Merkle tree}. The timestamp proves
that the transaction data existed when the block was published to get into its hash.

This structure allows for the datastructure to be immutable, and provides a linear forward history.
\section{Blockchain vs Distributed Ledger Technologies (DLT)}
Blockchain is a type of DLT i.e. not all DLTs are blockchain.

A DLT is a decentralized database that is managed byvarious participants. There is no central
authority that acts as monitor. Similar to a blockchain.

The main difference between Blockchain and DLT is that a Blockchain shares its records via blocks
, cryptograhpically protected blocks, i. e. a specific application of a DLT.

\chapter{Introduction to Bitcoin}
\section{Identity}
Identity is required for
\begin{itemize}
    \item Sending or Receiving money
    \item General accounting
\end{itemize}
In a similiar way as to a Home address and a mailbox key.
It's important to keep it secure.
\begin{itemize}
    \item Public key: is for receiving
    \item Private key: is for unlocking
    \item The private key is generated at Random once
    \item The public key is generated \textbf{from} the private key.
\end{itemize}
There a total of $2^{160}$ addresses possible. Each entity 
in Bitcoin is issued an address. The address is generated
from a hash of the users' public key. 
\section{Bitcoin Transactions}
\subsection{Distributed Database model}
\begin{itemize}
    \item Everyone stores a copy of the database
    \item Lightweight node: Only transaction headers are downloaded
    to validate transactions.
\end{itemize}

\section{Proof-of-work consensus}
\subsection{Bitcoin Security}
Example double spend attack.

We protect ourselves with timestamps.

\textbf{Blockchain Forking}: The longest chain is accepted as valid.
\section{Cryptography in Bitcoin}
\subsection{cryptograhpic hashing functions}
A function with these properties
\begin{itemize}
    \item Preimage resistance
    \item Second preimage resistance
    \item Collision resistance
\end{itemize}
\begin{theorem}
    \textbf{Preimage resistnace}: Calcualate the preimage of an output is almost imposible
\end{theorem}

\begin{theorem}
    \textbf{Second preimage resistance}: 
    $\forall x$ it is computationally difficult
    to find some value $x'\ \textit{such that } H(x)==H(x')$
\end{theorem}

\begin{theorem}
    \textbf{Collision resistance}: It is computationally difficult
    to find $x\land y\ \textit{such that } H(x)==H(y)$
\end{theorem}

Bitcoin uses the SHA-256$^2$ method hash function,
which consists of applying the SHA-256 to the output of 
a SHA-256: $\textit{SHA-}256(\textit{SHA-}256(x))$
\subsection{The Bitcoin block header}
A Bitcoin block consists of two parts \textbf{The header} and 
\textbf{The data}. Similar to an IP packet. 

The Header is composed of:
\begin{itemize}
    \item Previous block hash
    \item The Merkle root
    \item Nonce (Random number used only once)
\end{itemize}

\end{document}